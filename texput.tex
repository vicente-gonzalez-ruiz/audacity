\title{Audacity}

\maketitle
\tableofcontents

\chapter{Foundations}

\section{Features (and limitations)}

\begin{itemize}
\item Most common audio formats are supported (some of them directly
  such as Ogg Vorbis, WAV (Windows wave), FLAC (Free Lossless Audio
  Codec) or AIFF (Macinthosh Audio Interchange File Format) ... and
  others such as MP3 (by means of LAME) or AAC (by means of FFMPEG)).
\item Sample precission can be 16, 24 or 32 bits/sample. Therefore, it
  cannot capture 8 bits/sample and therefore, use logarithmic
  quantization. However, it can export these formats throught
  \href{http://www.mega-nerd.com/libsndfile/}{libsndfile}.
\item Several audio effects.
\item Expandable using plug-ins from
  \href{http://www.ladspa.org/}{LADSPA (Linux Audio Developer's Simple
    Plugin Api)} and writting plug-ins using
  \href{http://manual.audacityteam.org/o/man/nyquist_prompt.html}{the
    Nyquist programming language}.
\item Several analysis tools.
\item It is cross-platform and free
  (\href{http://audacity.sourceforge.net/about/license}{GPL})!
\end{itemize}

\section{Terminology}

\begin{itemize}
\item
  \href{http://manual.audacityteam.org/o/man/audacity_projects.html}{\textbf{Project}}:
  A collection of tracks, clips, labels, amplitude envelope points,
  gain and pan data.

\item \href{http://manual.audacityteam.org/o/man/tracks.html}{\textbf{Track}}:
  A sequence of audio samples.

\item
  \href{http://manual.audacityteam.org/o/man/audacity_tracks_and_clips.html}{\textbf{Clip}}:
  A pieze of a track.

\item
  \href{http://manual.audacityteam.org/o/man/label_tracks.html}{\textbf{Label}}:
  A name for a collection of samples of a track.

\item
  \href{http://manual.audacityteam.org/o/man/envelope_tool.html}{\textbf{Amplitude
      envelope point}}: A way to control the amplitude of a track
  along time.

\item
  \href{http://manual.audacityteam.org/o/man/glossary.html#gain}{\textbf{Gain}}:
  A measure of how much a signal is amplified. 0 dB means not
  amplification nor attenuation.

\item
  \href{http://manual.audacityteam.org/o/man/glossary.html#pan}{\textbf{Pan}}:
  Controls the spread of a stereo sound into the stereo channel.

\item
  \href{http://manual.audacityteam.org/o/man/sync_locked_track_groups.html}{\textbf{Sync-Locked
    Track Groups}}: lets you keep existing audio or labels synchronized
  with each other even when carrying out actions like inserting,
  deleting or changing speed or tempo.

\item \textbf{Stereo track}: a 2-channel track where each channel is a
  independent audio signal.

\item \textbf{Mono track}: a 2-channel track where there is only one
  audio signal (the same for each channel).

\item \textbf{Left/Right track}: a 1-channel track with data for, only, the
  left/right channel.

\item
  \href{http://manual.audacityteam.org/o/man/label_tracks.html}{\textbf{Label
    track}}: Used to reference points or regions in the project's audio
  tracks.

\item
  \href{http://manual.audacityteam.org/o/man/time_tracks.html}{\textbf{Time
    track}}: It is used in conjunction with one or more audio tracks to
  progressively increase or decrease playback speed (and pitch) over
  the length of the audio project (only one Time traks per project is
  allowed).

\item
  \href{http://manual.audacityteam.org/o/man/playing_and_recording.html}{\textbf{Playback
    Region}}: a track region associated with a shaded selection region
  in the waveform which is indicated in the timeline by a thin
  horizontal gray bar with arrowheads on each end.

\item
  \href{http://manual.audacityteam.org/o/man/faq_recording_how_to_s.html#overdub}{\textbf{Overdubbing}}:
  is the process of record sound while the existing tracks are played.

\item \textbf{Speed}: The number of samples/second played.

\item \textbf{Tempo}: It refeers to the playing time but not changing the pitch.

\item
  \href{http://manual.audacityteam.org/o/man/glossary.html#pitch}{\textbf{Pitch}}:
  The fundamental (first one) frequency of a sound (usually, a music
  note).

\end{itemize}

\chapter{\href{http://manual.audacityteam.org/o/man/index\_of\_effects\_generators\_and\_analyzers.html\#generators}{Synthesis (Generate Menu)}}

\section{\href{http://manual.audacityteam.org/o/man/generate_menu.html\#tone}{Tones}}
\begin{enumerate}
\item It is possible to generate pure tones (\verb|Generate->Tone|):
  \begin{enumerate}
  \item \textbf{Sine}: A mathematical sine function.
  \item \textbf{Square}: A pure periodic squared window
    function. Notice that this ``tone'' is physically impossible.
  \item \textbf{Square (no alias)}: Similar to the Square wave but
    does not produce aliasing distortion.
  \item \textbf{Sawtooth}: Has a gradual upwards slope followed by a shorter downwards slope. 
  \end{enumerate}
\item And time-decaying tones
  (\verb|Generate -> |\href{http://manual.audacityteam.org/o/man/generate_menu.html#chirp}{\texttt{Chirp}}). Using
  this option you can define a change in the frequency and the
  amplitude.
\end{enumerate}

\section{\href{http://manual.audacityteam.org/o/man/generate_menu.html\#dtmf}{Dual-Tone Multi-Frequency (DTMF) tones}}
\begin{itemize}
\item \verb|Generate -> DTMF Tones|.
\item For each tone you wish to generate, enter numbers from 0 to 9,
  lower case letters from a to z, and the * and \# characters. You can
  also enter the four ``priority'' tones used by the US Military
  (upper case A, B, C and D). Use the slider to select the ratio
  between the length of each tone in the series and the length of the
  silences between them.
\end{itemize}

\section{\href{http://manual.audacityteam.org/o/man/generate_menu.html\#noise}{Noise}}
\begin{itemize}
\item \verb|Generate -> Noise|.
\item Generates
  \href{http://en.wikipedia.org/wiki/White_noise}{white},
  \href{http://en.wikipedia.org/wiki/Pink_noise}{pink} and
  \href{http://en.wikipedia.org/wiki/Brownian_noise}{brownian} noise
  during a period of time. Notice that these noises have a different
  frequency composition.
\end{itemize}

\section{\href{http://manual.audacityteam.org/o/man/generate_menu.html\#silence}{Silence}}
\begin{itemize}
\item \verb|Generate -> Silence|.
\item Generates an ausence of signal during aperiod or time.
\end{itemize}

\section{\href{http://manual.audacityteam.org/o/man/click_track.html}{Clicks train}}
\begin{itemize}
\item \verb|Generate -> Click Track|.
\item Generates a sequence of regular pulses at the selected tempo.
\end{itemize}

\section{\href{http://manual.audacityteam.org/o/man/pluck.html}{Pluck}}
\begin{itemize}
\item \verb|Generate -> Pluck|.
\item Generates a synthesized pluck tone with abrupt or gradual
  fade-out, and selectable pitch corresponding to a MIDI note.
\end{itemize}

\section{\href{http://manual.audacityteam.org/o/man/risset_drum.html}{Risset Drum}}
\begin{itemize}
\item \verb|Generate -> Risset Drum|.
\item Produces a sound based on work by the composer of electronic
  music, Jean Claude Risset.
\end{itemize}

\chapter{Recording and playing}

\section{Sound driver selection}
\begin{itemize}
\item By default, in Linux, Audacity will use ALSA which is the best
  choice for avoiding latency issues during capturing audio.
\end{itemize}

\section{Input (recording) and output (playing) selection}
\begin{itemize}
\item Depending on your audio hardware, a number of input devices are
  available. Again, in Linux systems, ALSA native ``\texttt{hw}'' devices
  should minimize latency problems.
\end{itemize}

\section{\href{http://manual.audacityteam.org/o/man/transport_toolbar.html}{The Transport Toolbar}}
\begin{itemize}
\item The
  \href{http://manual.audacityteam.org/o/man/playing_and_recording.html}{Transport
    Toolbar} is the easiest way to control Audacity playback and
  recording. From left to right we found:
\begin{enumerate}
\item The \textbf{Pause} Button: press once to pause playback or
  recording then once to resume.
\item The \textbf{Play} Button: press once to start playback from the
  beggining.
\item The \textbf{Stop} Button: interrupts the playing/recording.
\item The \textbf{Skip to Start} Button: moves the time cursor to the
  start of the project.
\item The \textbf{Skip to End} Button: moves the time cursor to the
  end of the project.
\item The \textbf{Record} Button: start a new audio capture (you must
  push Stop to stop this process).
\end{enumerate}
\end{itemize}

\section{Using keyboard shorcuts}
\begin{enumerate}
\item \textbf{P}: Pause.
\item \textbf{SPACE}: Plays or stop.
\item \textbf{SHIFT+SPACE}: Plays in a loop.
\item \textbf{HOME}: Skip to Start.
\item \textbf{END}: Skip to End.
\item \textbf{R}: Record.
\item \textbf{SHIFT+R} or \textbf{SHIFT+Record}: Append-Record. Record
  starting from the end of the selected (previously existing)
  track(s).
\item \textbf{SHIFT+A}: Stop and set cursor. When stop a playback or a
  recording, the cursor or start of the selection is set to the
  position where playback/recording was stopped.
\end{enumerate}

\section{\href{http://manual.audacityteam.org/o/man/latency_test.html}{Measuring your latency}}
\begin{itemize}
\item Latency is an issue that depends mainly on your hardware (sound
  device and computer) and your audio buffer size %
  (\verb|Edit -> Preferences -> Recording -> Audio to buffer|).
\item In order to minimize the latency, you can improve your hardware
  and/or reduce the audio buffer size. However, reducing the audio
  buffer size can produce underruns in the capturing process if the
  CPU is not able to manage this configuration requirement.
\item When overdubbing (see Section~\ref{sec:overdubbing}), you must
  measure and set your current lantency in %
  \verb|Edit -> Preferences -> Recording -> Latency Correction|,
  otherwise those audio tracks that have been previously captured will
  be the synchronized with the new ones.
\item To find out your latency you must capture a sound produced by
  your computer and compute the delay between the time the sound has
  been produced and the time the sound has been captured. To do this:
  \begin{enumerate}
  \item In Selection Toolbar make sure that "\verb|Snap To|" is set to
    "\verb|Off|".
  \item Above the second group of numbers, make sure that
    ``\verb|Length|'' is selected.
  \item Click on one of the downward-pointing arrows in the digits
    boxes to the right of ``\verb|Snap To|'' and select
    ``\verb|hh:mm:ss + milliseconds|''.
  \item Generate 2 bars of click track %
    (\verb|Generate -> Click Track|), and choose the default
    ``\verb|Ping|'' sound.
  \item Click ``\verb|OK|'' to generate the click track.
  \item Now click the Record Button in the Transport Toolbar.
  \item You will get a new track. The top track is the original click
    track, the bottom track is the looped-back recording.
  \item Zoom in so you can see one of the clicks in the top track and
    its delayed version on the bottom track.
  \item You can now read the latency directly from the second panel of
    numbers.
  \item Write in the
    \verb|Edit -> Preferences -> Recording tab -> Latency Correction|
    Entry the negative of this number.
  \item Repeat this experiment to check if the latency when
    overdubbing has been ``hidden''.
  \end{enumerate}
\end{itemize}

\section{Timer recording}
\begin{itemize}
\item The \verb|Transport -> Timer Record| Dialog allows to configure
  a temporized recording (starting date and time and duration).
\end{itemize}

\section{Playing slower/faster}
\begin{itemize}
\item Use the
  \href{http://manual.audacityteam.org/o/man/transcription_toolbar.html}{Transcription
    Toolbar} to change the playing speed.
\end{itemize}

\section{\href{http://manual.audacityteam.org/o/man/time_tracks.html}{Time warping}}
\begin{itemize}
\item Progressively increase or decrease playback speed (and pitch).
\end{itemize}

\section{Overdubbing}
\label{sec:overdubbing}
\begin{itemize}
\item By default, Audacity record and plays tracks simultaneously. See %
  \verb|Transport -> Overdub (on/off)|.
\end{itemize}

\chapter{\href{http://manual.audacityteam.org/o/man/index_of_effects_generators_and_analyzers.html\#analyzers}{Analysis (Analyze Menu)}}

\section{\href{http://manual.audacityteam.org/o/man/contrast.html}{Contrast}}
\begin{itemize}
\item \verb|Analyze -> Contrast|.
\item Can be used to compare the audio level of two different parts of
  a mono track. Suppossing that one of the parts is the
  \emph{foreground} (that obviously also has a background noise) and
  the other is the \emph{backgroud} (with only background noise), this
  analysis can be useful to know if there is enoung signal (20 dB or
  more) in the foreground section compared to the background section.
\item Procedure:
  \begin{enumerate}
  \item Select a region containing the signal (speech, for
    example). This is the "foreground" selection.
  \item Click the ``\verb|Foreground|'''s ``\verb|Measure selection|'' Button.
  \item Select a region containing only the background sound. This is
    the background selection.
  \item Click the ``\verb|Background|'''s ``\verb|Measure selection|'' Button.
  \end{enumerate}
\end{itemize}

\section{\href{http://manual.audacityteam.org/o/man/find_clipping.html}{Clipping ocurrency}}
\begin{itemize}
\item \verb|Analyze -> Find Clipping|.
\item Displays runs of clipped samples in a Label Track.
\item You can control the length of the clipped runs (default 3
  samples) and the number of unclipped samples (default also 3) that
  must occur before a sun of clipped samples will be determined.
\end{itemize}

\section{\href{http://manual.audacityteam.org/o/man/plot_spectrum.html}{Frequency analysis}}
\begin{itemize}
\item \verb|Analyze -> Plot spectrum|.
\item Graphs the spectral density function of the sound.
\item You can choose between:
  \begin{enumerate}
  \item \textbf{Spectral analysis}: A genuine Fourier analysis.
  \item \textbf{Autocorrelation}: A measure of the redundancy of the signal.
  \item \textbf{Cepstrum}: The power cepstrum $C$ (of a signal) is the
    squared magnitude of the Fourier transform of the logarithm of the
    squared magnitude of the Fourier transform of a
    signal. Mathematically:
    \begin{equation*}
      C=|{\cal{F}}^{-1}(\log_2(|{\cal{F}}|^2)))|^2
    \end{equation*}
    Physically, the cepstum resents the rate of change in the
    different spectrum bands. It's particularly useful for properties
    of vocal tracks and is used, for example, to identify speakers by
    their voice characteristics.
  \end{enumerate}
\end{itemize}

\section{\href{http://manual.audacityteam.org/o/man/beat_finder.html}{Localize beats}}
\begin{itemize}
\item \verb|Analyze -> Beat Finder|.
\item Place labels at beats which are much louder than the surrounding
  audio.
\end{itemize}

\section{\href{http://manual.audacityteam.org/o/man/regular_interval_labels.html}{Track splitting}}
\begin{itemize}
\item \verb|Analyze -> Regular Interval Labels|.
\item Places point labels in a label track so as to divide the
  associated audio into smaller, equally-sized segments.
\end{itemize}

\section{\href{http://manual.audacityteam.org/o/man/silence_finder_and_sound_finder.html\#silence}{Silence finder}}
\begin{itemize}
\item \verb|Analyze -> Silence Finder|.
\item Divides up a track by placing point labels inside areas of silence.
\end{itemize}

\section{\href{http://manual.audacityteam.org/o/man/silence_finder_and_sound_finder.html\#sound}{Sound finder}}
\begin{itemize}
\item \verb|Analyze -> Sound Finder|.
\item Divides up a track by placing region labels for areas of sound
  that are separated by silence.
\end{itemize}

\chapter{Editing (Edit View and Tracks Menus)}

\section{Creating a new tracks}
\begin{itemize}
\item \textbf{Mono track}: go to %
  \verb|Tracks -> Add new -> Audio Track|.
\item \textbf{Stereo track}: go to %
  \verb|Tracks -> Add new -> Stereo Track|.
\item \textbf{Label track}: go to % 
  \verb|Tracks -> Add new -> Label Track|.
\end{itemize}

\section{\href{http://manual.audacityteam.org/o/man/splitting_and_joining_stereo_tracks.html}{Splitting and joining tracks}}
\begin{itemize}
\item Using the
  \href{http://manual.audacityteam.org/o/man/track_drop_down_menu.html}{Track
    Drop-Down menu} you can:
  \begin{enumerate}
  \item Split a stereo track into separate tracks for left and right
    channels after choosing
    \verb|Track Drop-Down -> Split Stereo Track|.
  \item Split a stereo track into two separate mono tracks after
    selecting
    \verb|Track Drop-Down -> Stereo to Mono|.
  \item Swap the channels in a stereo track.
  \item
    \href{http://manual.audacityteam.org/o/man/audio_tracks.html}{Join
      two mono, left or right tracks into one stereo track}, by
    selecting \verb|Track Drop-Down -> Make Stereo Track|..
  \end{enumerate}
\end{itemize}

\section{\href{http://manual.audacityteam.org/o/man/tracks_menu.html\#mix}{Mixing tracks}}
\begin{itemize}
\item When you reproduce (push the Play Button) a project, all tracks
  are mixed together in order to play them, but you don't have a track
  that can be written into a stereo audio file with the result of our
  project.
\item The \verb|Mix and Render| Option of the \verb|Tracks| Menu
  explicitly mixes down all selected tracks to a single mono or stereo
  track.  The resulting track (called ``\verb|Mix|'') replaces the
  selected tracks and is placed underneath any tracks that were not
  mixed and rendered.
\item It is possible also to mix and render to a new track by using
  CTRL+SHIFT+M.
\end{itemize}

\section{\href{http://manual.audacityteam.org/o/man/zooming.html}{Zooming the time scale}}
\begin{itemize}
\item To zoom in, position the mouse pointer over a track and
  left-click. To zoom out, shift-click or click the right mouse
  button.
\end{itemize}

\section{\href{http://manual.audacityteam.org/o/man/zooming.html}{Zooming the amplitude scale}}
\begin{itemize}
\item Hover the mouse over the vertical ruler of a track, and the
  pointer changes to a magnifying glass, indicating you can zoom
  vertically.
\end{itemize}

%\section{Track aligning}

\section{Resampling}

\begin{itemize}
\item Resampling is usually used to match a given audio format (such
  as the CD quality audio).
\item However, resampling increments the number of samples (and
  therefore the amount of memory) but not increases the quality of the
  sound.
\end{itemize}

\subsection{\href{http://manual.audacityteam.org/o/man/tracks_menu.html\#mix}{Project resampling}}
\begin{itemize}
\item \verb|Tracks -> Resample| ...  and select the new sampling rate.
\end{itemize}

\subsection{Track resampling}
\begin{itemize}
\item \verb|Audio Track Pop-up -> Set Rate| ... and select the new sampling rate.
\end{itemize}

\section{\href{http://manual.audacityteam.org/o/man/repeat.html}{Repeat}}
\begin{itemize}
\item \verb|Effect -> Repeat|.
\item Repeats the selection the specified number of times. 
\end{itemize}

\chapter{Built-in Amplitude effects (Effects Menu)}
%\begin{itemize}
%\item Make the sound louder or quieter
%\end{itemize}

\section{\href{http://manual.audacityteam.org/o/man/amplify.html}{Scaling (amplification/atenuation)}}
\begin{itemize}
\item \verb|Effect -> Amplify -> Amplification (dB)|.
\item Similar to move the potentiometer of your amplifier.
\item In a stereo track, the balance between the left and right
  channels will be preserved.
\end{itemize}

\section{\href{http://manual.audacityteam.org/o/man/normalize.html}{Normalization}}
\begin{itemize}
\item \verb|Effect -> Normalize -> Normalize maximum amplitude to|.
\item Similar to
  \href{http://manual.audacityteam.org/o/man/amplify_and_normalize.html}{scaling},
  but removes the DC offset.
\item In a stereo track, the balance between the left and right
  channels can be preserved (uncheck Normalize stereo channel
  independetly).
\end{itemize}

\section{\href{http://manual.audacityteam.org/o/man/fades.html\#linearfade}{Fadding}}
\begin{itemize}
\item \texttt{Effect -> Fade In | Fade Out |
    \href{http://manual.audacityteam.org/o/man/adjustable_fade.html}{Adjustable
      Fade}}.
\item Produces smooth starts and finishes in the audio tracks. 
\end{itemize}

\section{\href{http://manual.audacityteam.org/o/man/auto_duck.html}{Temporal amplitude attenuation (duck)}}
\begin{itemize}
\item \verb|Effect -> Duck|.
\item Reduces (ducks) the volume of one or more selected tracks
  whenever the volume of a specified "control track" reaches a
  particular threshold level.
\end{itemize}

\section{\href{http://manual.audacityteam.org/o/man/compressor.html}{Compression}}
\begin{itemize}
\item \verb|Effect -> Compressor|.
\item Reduces the dynamic range of audio, ampliying the quiet sounds.
\end{itemize}

\section{\href{http://manual.audacityteam.org/o/man/hard_limiter.html}{Hard Limiter}}
\begin{itemize}
\item \verb|Effect -> Hard Limiter|.
\item Such as the Compression Effect, although clipping can happen.
\end{itemize}

\section{\href{http://manual.audacityteam.org/o/man/leveler.html}{Leveler}}
\begin{itemize}
\item \verb|Effect -> Leveler|.
\item A simple, combined compressor and limiter effect for reducing
  the dynamic range of audio. It reduces the difference between loud
  and soft, making the audio easier to hear in noisy environments or
  on small loudspeakers.
\end{itemize}

\section{\href{http://manual.audacityteam.org/o/man/sc4.html}{SC4}}
\begin{itemize}
\item \verb|Effect -> SC4|.
\item Another stereo compressor.
\end{itemize}

%\chapter{Quality effects}
%\begin{itemize}
%\item Change the quality of the sound.
%\end{itemize}

\section{\href{http://manual.audacityteam.org/o/man/bass_and_treble.html}{Bass and Treble}}
\begin{itemize}
\item \verb|Effect -> Bass and Treble|.
\item Increases or decreases the lower frequencies and higher
  frequencies of your audio independently.
\end{itemize}

\section{\href{http://manual.audacityteam.org/o/man/change_pitch.html}{Change Pitch}}
\begin{itemize}
\item \verb|Effect -> Change Pitch|.
\item Change the pitch of a track without changing its tempo (speed).
\end{itemize}

\section{\href{http://manual.audacityteam.org/o/man/equalization.html}{Equalization}}
\begin{itemize}
\item \verb|Effect -> Equalization|.
\item Allows you to increase the volume of some frequencies and reduce others.
\end{itemize}

\section{\href{http://manual.audacityteam.org/o/man/high_pass_filter.html}{High Pass Filter}}
\begin{itemize}
\item \verb|Effect -> High Pass Filter|.
\item Removes frequencies above the cutoff frequency.
\end{itemize}

\section{\href{http://manual.audacityteam.org/o/man/low_pass_filter.html}{Low Pass Filter}}
\begin{itemize}
\item \verb|Effect -> Low Pass Filter|.
\item Removes frequencies below the cutoff frequency.
\end{itemize}

% Tal vez se puede mover a otro capitulo
\section{\href{http://manual.audacityteam.org/o/man/paulstretch.html}{Paulstretch}}
\begin{itemize}
\item \verb|Effect -> Paulstretch|.
\item Slow down audio without changing the pitch.
\end{itemize}

\section{\href{http://manual.audacityteam.org/o/man/phaser.html}{Phaser}}
\begin{itemize}
\item \texttt{Effect -> \href{http://en.wikipedia.org/wiki/Phaser\_(effect)}{Phaser}}.
\item Displaces frequencies of the audio and the displacement is
  controlled by a low frequency sine signal.
\end{itemize}

\section{\href{http://manual.audacityteam.org/o/man/tremolo.html}{Tremolo}}
\begin{itemize}
\item \verb|Effect -> Tremolo|.
\item Is the variation of the amplitude of the sound controlled by a
  sine signal (amplitude modulation).
\end{itemize}

\section{\href{http://manual.audacityteam.org/o/man/vocoder.html}{Vocoder}}
\begin{itemize}
\item \verb|Effect -> Vocoder|.
\item Modulates the left channel of a stereo track with the right channel and viceversa.
\end{itemize}

\section{\href{http://manual.audacityteam.org/o/man/wahwah.htm}{Wahwah}}
\begin{itemize}
\item \verb|Effect -> Wahwah|.
\item Uses a moving (back and forth in the frequency) bandpass filter
  to create its sound controlled by a low frequency oscillator.
\end{itemize}

\section{\href{http://manual.audacityteam.org/o/man/click_removal.html}{Declicking}}
\begin{itemize}
\item \verb|Effect -> Click Removal|.
\item Remove clicks (especially suited for those procuced by vinyl records) without damaging the rest of the audio.
\end{itemize}

\section{\href{http://manual.audacityteam.org/o/man/clip_fix.html}{Reconstruct clipped regions}}
\begin{itemize}
\item \verb|Effect -> Clip Fix|.
\item Reconstruct clipped regions by interpolating the lost signal.
\end{itemize}

\section{\href{http://manual.audacityteam.org/o/man/noise_removal.html}{Noise Removal}}
\begin{itemize}
\item \verb|Effect -> Noise Removal|.
\item Reduce constant background sounds such as ``hum'' (zumbido),
  ``whistle'' (silvido), ``whine'' (gemido) or ``buzz'' (rumor), and
  moderate amounts of ``hiss'' (siseo).
\end{itemize}

\section{\href{http://manual.audacityteam.org/o/man/notch_filter.html}{Notch Filter}}
\begin{itemize}
\item \verb|Effect -> Notch Filter|.
\item Sharply attenuates frequency-specific noise like mains hum or
  electrical whistle with minimal damage to the remaining audio.
\end{itemize}

\section{\href{http://manual.audacityteam.org/o/man/repair.html}{Repair}}
\begin{itemize}
\item \verb|Effect -> Repair|.
\item Removes a very short region (up to 128 samples) of damaged or
  destroyed audio, replacing it with an estimated region of audio
  based on what is happening either side of the region.
\end{itemize}

\section{\href{http://manual.audacityteam.org/o/man/truncate_silence.html}{Truncate Silence}}
\begin{itemize}
\item \verb|Effect -> Truncate Silence|.
\item Reduces the length of passages where the volume is below a specified level.
\end{itemize}

\section{\href{http://manual.audacityteam.org/o/man/change_tempo.html}{Change Tempo}}
\begin{itemize}
\item \verb|Effect -> Change Tempo|.
\item Change the tempo of the selection without changing its pitch.
\end{itemize}

\section{\href{http://manual.audacityteam.org/o/man/change_speed.html}{Change Speed}}
\begin{itemize}
\item \verb|Effect -> Change Speed|.
\item Change the speed of a track, affecting its tempo, pitch and
  frequency content.
\end{itemize}

\section{\href{http://manual.audacityteam.org/o/man/sliding_time_scale_pitch_shift.html}{Sliding Time Scale / Pitch Shift}}
\begin{itemize}
\item \verb|Effect -> Sliding Time Scale| / verb|Pitch Shift|.
\item Make a continuous change to the tempo and/or pitch of a
  selection by choosing initial and/or final change values.
\end{itemize}

\section{\href{http://manual.audacityteam.org/o/man/delay.html}{Delay}}
\begin{itemize}
\item \verb|Effect -> Delay|.
\item A multiple echo effect.
\end{itemize}

\section{\href{http://manual.audacityteam.org/o/man/echo.html}{Echo}}
\begin{itemize}
\item \verb|Effect -> Echo|.
\item Repeats the audio you have selected again and again, normally softer each time.
\end{itemize}

\section{\href{http://manual.audacityteam.org/o/man/reverb.html}{Reverb}}
\begin{itemize}
\item \verb|Effect -> Reverb|.
\item Adds reverberation (rapid, modified repetitions blended with the
  original sound that gives an impression of ambience).
\end{itemize}

\section{\href{http://manual.audacityteam.org/o/man/invert.html}{Invert}}
\begin{itemize}
\item \verb|Effect -> Invert|.
\item Flips the audio samples upside-down, reversing their
  polarity. Notice that if you play the original and the inverted
  audio together, both tracks will be substracted.
\end{itemize}

\section{\href{http://manual.audacityteam.org/o/man/reverse.html}{Reverse}}
\begin{itemize}
\item \verb|Effect -> Reverse|.
\item Reverses the selected audio.
\end{itemize}

\section{\href{http://manual.audacityteam.org/o/man/nyquist_prompt.html}{Nyquist Prompt}}
\begin{itemize}
\item \verb|Effect -> Nyquist Prompt|.
\item Write your own plug-ins created using the Nyquist programming language.
\end{itemize}
